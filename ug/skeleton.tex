% UG project example file, February 2024
%
%   Added the "online" option for equal margins, February 2024 [Hiroshi Shimodaira, Iain Murray]
%   A minor change in citation, September 2023 [Hiroshi Shimodaira]
%
% Do not change the first two lines of code, except you may delete "logo," if causing problems.
% Understand any problems and seek approval before assuming it's ok to remove ugcheck.
\documentclass[logo,bsc,singlespacing,parskip,online]{infthesis}
\usepackage{ugcheck}


% Include any packages you need below, but don't include any that change the page
% layout or style of the dissertation. By including the ugcheck package above,
% you should catch most accidental changes of page layout though.

\usepackage{microtype} % recommended, but you can remove if it causes problems
\usepackage[round]{natbib} % recommended for citations

% --- Colors & code formatting ---
\usepackage{xcolor}
\usepackage{listings}

% Define colors safely
\definecolor{codebg}{rgb}{0.97,0.97,0.97}
\definecolor{ocamlblue}{rgb}{0.2,0.2,0.6}
\definecolor{gray}{rgb}{0.4,0.4,0.4}
\definecolor{orange}{rgb}{1.0,0.55,0.0}

% Define OCaml syntax highlighting
\lstdefinelanguage{OCaml}{
  keywords={let, rec, match, with, type, module, open, effect, perform, continue, and, in, fun, function},
  keywordstyle=\color{ocamlblue}\bfseries,
  comment=[s]{(*}{*)},
  commentstyle=\color{gray}\ttfamily,
  stringstyle=\color{orange},
  basicstyle=\ttfamily\small,
  backgroundcolor=\color{codebg},
  frame=single,
  breaklines=true,
  showstringspaces=false
}

\begin{document}
\begin{preliminary}

\title{A Probabilistic Inference Engine Utilizing Effect Handlers}

\author{Arindam Bag}

% CHOOSE YOUR DEGREE a):
% please leave just one of the following un-commented
\course{Artificial Intelligence and Computer Science}
%\course{Artificial Intelligence and Computer Science}
%\course{Artificial Intelligence and Mathematics}
%\course{Artificial Intelligence and Software Engineering}
%\course{Cognitive Science}
%\course{Computer Science}
%\course{Computer Science and Management Science}
%\course{Computer Science and Mathematics}
%\course{Computer Science and Physics}
%\course{Software Engineering}
%\course{Master of Informatics} % MInf students

% CHOOSE YOUR DEGREE b):
% please leave just one of the following un-commented
%\project{MInf Project (Part 1) Report}  % 4th year MInf students
%\project{MInf Project (Part 2) Report}  % 5th year MInf students
\project{4th Year Project Report}        % all other UG4 students


\date{\today}

\abstract{
This skeleton demonstrates how to use the \texttt{infthesis} style for
undergraduate dissertations in the School of Informatics. It also emphasises the
page limit, and that you must not deviate from the required style.
The file \texttt{skeleton.tex} generates this document and should be used as a
starting point for your thesis. Replace this abstract text with a concise
summary of your report.
}

\maketitle

\newenvironment{ethics}
   {\begin{frontenv}{Research Ethics Approval}{\LARGE}}
   {\end{frontenv}\newpage}

\begin{ethics}

This project was planned in accordance with the Informatics Research
Ethics policy. It did not involve any aspects that required approval
from the Informatics Research Ethics committee.

\standarddeclaration
\end{ethics}


\begin{acknowledgements}
Any acknowledgements go here.
\end{acknowledgements}


\tableofcontents
\end{preliminary}


\chapter{Introduction}

This chapter introduces the motivation behind effect-based probabilistic programming and explains the rationale for modeling inference using algebraic effects and handlers. In addition, it outlines the objectives and contributions of the project and presents the structure of the report.

\section{Motivation for Effects in Probabilistic Programming}

Probabilistic programming languages (PPLs) provide a high-level abstraction for expressing statistical models while delegating inference to a general-purpose inference engine. This separation allows users to focus on model specification rather than algorithmic details. 

Many inference algorithms such as Metropolis-Hastings, Particle Filters, and their hybrid variants can be viewed as different execution strategies over the same underlying probabilistic model. The algorithms perform the same fundamental probabilistic operations and target the same posterior distribution, differing only in how they explore it.

In practice, the design of inference mechanisms in many PPLs tightly couples model execution with inference logic. For example, \textbf{Gen} requires models to be explicitly parameterized with execution structure, requiring the user to expose execution steps to support certain inference algorithms. \textbf{Pyro} requires a specialized method for each model to support sequential inference algorithms. Other languages such as \textbf{Anglican} use continuation-passing-style transformations to expose control over execution, but this control is obtained through global program rewriting instead of modular handlers, making inference logic harder to isolate and reuse.


Meng et al. addresses this limitation by formulating probabilistic programming in terms of algebraic effects and handlers. Their system, \textbf{ProbFX}, expresses probabilistic operations such as sampling, observations, and resampling as effects, while inference algorithms are expressed as effect handlers. This formulation makes the separation between model and inference explicit. A single model definition can be interpreted by multiple inference handlers, each implementing a different strategy for exploring the posterior distribution without requiring any modification to the model itself.

(PLACEHOLDER ASK DANIEL about scoping contributions)

While ProbFX demonstrates the expressive power of algebraic effects for probabilistic inference, questions remain about how these ideas transfer across languages and how alternative effect-based designs impact modularity and performance.







\section{Project Aims}

\section{Project Contributions}

\section{Report Structure}

The remaining chapters are 

Background

Implementation
    - Section for Immutable Traces
    - Section for Multicont

Testing Suite

Evaluation and Reflections

Conclusion and Future Work

\chapter{Background}

\section{Effects and Handlers}

Effect handler-oriented programming is a programming paradigm that helps with the creation of composable code. Effects are able to extend exception handling by enabling continuation of a program halted by the detection of an exception. It is also possible to pass values to the program after the continuation. Keeping this in mind, you can implement multiple computational effects in code with this paradigm like logging, non-determinism, and state management.

An effect is defined by its effect signature [Effects are interfaces], which defines the name of the effect in scope. It also specifies the type of the input and output of the effect. This serves as an instruction for the programmer when they design the handlers for these effects. 

Handlers are implementations for a particular effect. Let us say you have an effect called \texttt{Crop} [CHANGE THIS TO ONE OF THE EFFECTS I WROTE BELOW]. There are multiple ways to crop an image, such as rectangular, free-form, and circular. Provided that you keep the types of handlers consistent with what has been defined in the effect signature, you will be able to implement any number of crop operations. 

These handlers enable a high degree of composabality because of continuations. You can pass a value to a continuation that can invoke another handler, and so on. This creates a modular and composable style of code built on Effect Handlers. 

[Bit on Mathematical Foundations of Effects]

[https://dhil.net/research/papers/thesis.pdf]

\section{Examples in Code}

This section will highlight some practical examples of effect handlers in code. 

I have chosen to use OCaml to represent these examples, and to write the dissertation. OCaml has a rich effects system and is a functional language with strong pattern matching capabilities. Its support for effects and handlers provides a natural foundation for expressing probabilistic computations as composable and interpretable effects.

\subsection{Basic Components of an Effect Oriented Program}


\subsubsection{{Handling the \texttt{Choose} Effect}}

The \texttt{Choose} effect represents an operation where the program must select one value from a list of items. The actual choice depends on the effect handler, which determines the behavior of this effect.

\subsubsection{Random Choice Handler}

This handler initializes the random number generator, captures the continuation \texttt{k}, and resumes execution with a randomly chosen element from the list.


\begin{lstlisting}[language=OCaml]
open Effect
open Effect.Deep
open Random

type _ Effect.t += Choose : int list -> int Effect.t

let handle_random f =
  Random.self_init ();
  match_with f ()
  { retc = (fun v -> v);
    exnc = (fun e -> raise e);
    effc = (fun (type a) (eff : a t) ->
      match eff with
      | Choose lst ->
          Some (fun (k : (a, _) continuation) ->
            let idx = Random.int (List.length lst) in
            let choice = List.nth lst idx in
            continue k choice)
      | _ -> None)
  }
\end{lstlisting}

\subsubsection{Deterministic Choice Handler}

Alternatively, we can define a deterministic handler that always selects the element at a specific index. This is useful for reproducibility or debugging.

\begin{lstlisting}[language=OCaml]
let handle_indexed index f =
  match_with f ()
  { retc = (fun v -> v);
    exnc = (fun e -> raise e);
    effc = (fun (type a) (eff : a t) ->
      match eff with
      | Choose lst ->
          Some (fun (k : (a, _) continuation) ->
            if index < List.length lst then
              continue k (List.nth lst index)
            else failwith "Index out of range")
      | _ -> None)
  }
\end{lstlisting}

\subsubsection{Demonstration Program}

We can now define a small program that performs the \texttt{Choose} effect twice and evaluates it under both handlers.

\begin{lstlisting}[language=OCaml]
let choose_demo () =
  let x = perform (Choose [1; 2; 3]) in
  let y = perform (Choose [10; 20; 30]) in
  x + y

let () =
  Printf.printf "Random result: %d\n" (handle_random choose_demo);
  Printf.printf "Indexed result: %d\n" (handle_indexed 0 choose_demo);
\end{lstlisting}

When run, the program produces different results depending on the chosen handler. The random handler generates variable results, while the indexed handler provides a single fixed result each time.

\subsection{Prior Work}

Automatic Differentiation via Effects and Handlers
JESSE SIGAL

HANDLING ALGEBRAIC EFFECTS
GORDON D. PLOTKIN AND MATIJA PRETNAR


\subsection{Probability}



\chapter{Conclusions}

\section{Final Reminder}

The body of your dissertation, before the references and any appendices,
\emph{must} finish by page~40. The introduction, after preliminary material,
should have started on page~1.

You may not change the dissertation format (e.g., reduce the font size, change
the margins, or reduce the line spacing from the default single spacing). Be
careful if you copy-paste packages into your document preamble from elsewhere.
Some \LaTeX{} packages, such as \texttt{fullpage} or \texttt{savetrees}, change
the margins of your document. Do not include them!

Over-length or incorrectly-formatted dissertations will not be accepted and you
would have to modify your dissertation and resubmit. You cannot assume we will
check your submission before the final deadline and if it requires resubmission
after the deadline to conform to the page and style requirements you will be
subject to the usual late penalties based on your final submission time.

% \bibliographystyle{plain}
\bibliographystyle{plainnat}
\bibliography{mybibfile}


% You may delete everything from \appendix up to \end{document} if you don't need it.
\appendix

\chapter{First appendix}

\section{First section}

Any appendices, including any required ethics information, should be included
after the references.

Markers do not have to consider appendices. Make sure that your contributions
are made clear in the main body of the dissertation (within the page limit).

\chapter{Participants' information sheet}

If you had human participants, include key information that they were given in
an appendix, and point to it from the ethics declaration.

\chapter{Participants' consent form}

If you had human participants, include information about how consent was
gathered in an appendix, and point to it from the ethics declaration.
This information is often a copy of a consent form.


\end{document}
