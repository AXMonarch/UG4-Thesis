% UG project example file, February 2024
%
%   Added the "online" option for equal margins, February 2024 [Hiroshi Shimodaira, Iain Murray]
%   A minor change in citation, September 2023 [Hiroshi Shimodaira]
%
% Do not change the first two lines of code, except you may delete "logo," if causing problems.
% Understand any problems and seek approval before assuming it's ok to remove ugcheck.
\documentclass[logo,bsc,singlespacing,parskip,online]{infthesis}
\usepackage{ugcheck}


% Include any packages you need below, but don't include any that change the page
% layout or style of the dissertation. By including the ugcheck package above,
% you should catch most accidental changes of page layout though.

\usepackage{microtype} % recommended, but you can remove if it causes problems
\usepackage[round]{natbib} % recommended for citations

% --- Colors & code formatting ---
\usepackage{xcolor}
\usepackage{listings}

% Define colors safely
\definecolor{codebg}{rgb}{0.97,0.97,0.97}
\definecolor{ocamlblue}{rgb}{0.2,0.2,0.6}
\definecolor{gray}{rgb}{0.4,0.4,0.4}
\definecolor{orange}{rgb}{1.0,0.55,0.0}

% Define OCaml syntax highlighting
\lstdefinelanguage{OCaml}{
  keywords={let, rec, match, with, type, module, open, effect, perform, continue, and, in, fun, function},
  keywordstyle=\color{ocamlblue}\bfseries,
  comment=[s]{(*}{*)},
  commentstyle=\color{gray}\ttfamily,
  stringstyle=\color{orange},
  basicstyle=\ttfamily\small,
  backgroundcolor=\color{codebg},
  frame=single,
  breaklines=true,
  showstringspaces=false
}

\begin{document}
\begin{preliminary}

\title{A Probabilistic Inference Engine Utilizing Effect Handlers}

\author{Arindam Bag}

% CHOOSE YOUR DEGREE a):
% please leave just one of the following un-commented
\course{Artificial Intelligence and Computer Science}
%\course{Artificial Intelligence and Computer Science}
%\course{Artificial Intelligence and Mathematics}
%\course{Artificial Intelligence and Software Engineering}
%\course{Cognitive Science}
%\course{Computer Science}
%\course{Computer Science and Management Science}
%\course{Computer Science and Mathematics}
%\course{Computer Science and Physics}
%\course{Software Engineering}
%\course{Master of Informatics} % MInf students

% CHOOSE YOUR DEGREE b):
% please leave just one of the following un-commented
%\project{MInf Project (Part 1) Report}  % 4th year MInf students
%\project{MInf Project (Part 2) Report}  % 5th year MInf students
\project{4th Year Project Report}        % all other UG4 students


\date{\today}

\abstract{
This skeleton demonstrates how to use the \texttt{infthesis} style for
undergraduate dissertations in the School of Informatics. It also emphasises the
page limit, and that you must not deviate from the required style.
The file \texttt{skeleton.tex} generates this document and should be used as a
starting point for your thesis. Replace this abstract text with a concise
summary of your report.
}

\maketitle

\newenvironment{ethics}
   {\begin{frontenv}{Research Ethics Approval}{\LARGE}}
   {\end{frontenv}\newpage}

\begin{ethics}
\textbf{Instructions:} \emph{Agree with your supervisor which
statement you need to include. Then delete the statement that you are not using,
and the instructions in italics.\\
\textbf{Either complete and include this statement:}}\\ % DELETE THESE INSTRUCTIONS
%
% IF ETHICS APPROVAL WAS REQUIRED:
This project obtained approval from the Informatics Research Ethics committee.\\
Ethics application number: ???\\
Date when approval was obtained: YYYY-MM-DD\\
%
\emph{[If the project required human participants, edit as appropriate, otherwise delete:]}\\ % DELETE THIS LINE
The participants' information sheet and a consent form are included in the appendix.\\
%
% IF ETHICS APPROVAL WAS NOT REQUIRED:
\textbf{\emph{Or include this statement:}}\\ % DELETE THIS LINE
This project was planned in accordance with the Informatics Research
Ethics policy. It did not involve any aspects that required approval
from the Informatics Research Ethics committee.

\standarddeclaration
\end{ethics}


\begin{acknowledgements}
Any acknowledgements go here.
\end{acknowledgements}


\tableofcontents
\end{preliminary}


\chapter{Introduction}

The preliminary material of your report should contain:
\begin{itemize}
\item
The title page.
\item
An abstract page.
\item
Declaration of ethics and own work.
\item
Optionally an acknowledgements page.
\item
The table of contents.
\end{itemize}

As in this example \texttt{skeleton.tex}, the above material should be
included between:
\begin{verbatim}
\begin{preliminary}
    ...
\end{preliminary}
\end{verbatim}
This style file uses roman numeral page numbers for the preliminary material.

The main content of the dissertation, starting with the first chapter,
starts with page~1. \emph{\textbf{The main content must not go beyond page~40.}}

The report then contains a bibliography and any appendices, which may go beyond
page~40. The appendices are only for any supporting material that's important to
go on record. However, you cannot assume markers of dissertations will read them.

You may not change the dissertation format (e.g., reduce the font size, change
the margins, or reduce the line spacing from the default single spacing). Be
careful if you copy-paste packages into your document preamble from elsewhere.
Some \LaTeX{} packages, such as \texttt{fullpage} or \texttt{savetrees}, change
the margins of your document. Do not include them!

Over-length or incorrectly-formatted dissertations will not be accepted and you
would have to modify your dissertation and resubmit. You cannot assume we will
check your submission before the final deadline and if it requires resubmission
after the deadline to conform to the page and style requirements you will be
subject to the usual late penalties based on your final submission time.

\section{Using Sections}

Divide your chapters into sub-parts as appropriate.

\section{Citations}

When citing work using author names, your sentences should still read
as correct English if any parts in parenthesis are removed.
Use {\tt {\textbackslash}citet} when citing the name as text that's part of your sentence: 
\begin{quote}
  We follow the method of \citet{P1}.
\end{quote}
and use {\tt {\textbackslash}citep} for a parenthetical citation:
\begin{quote}
  It's possible to learn first-order Horn sequences from entailment \citep{P2}.
\end{quote}

You may use any consistent reference style that you prefer, including ``[1]'' numerical citations. 

\chapter{Background}

\section{Effects and Handlers}

Effect handler-oriented programming is a programming paradigm that helps with the creation of composable code. Effects are able to extend exception handling by enabling continuation of a program halted by the detection of an exception. It is also possible to pass values to the program after the continuation. Keeping this in mind, you can implement multiple side effects with this paradigm. [Examples]

An effect is defined by its effect signature, which defines the name of the effect in scope. It also specifies the type of the input and output of the effect. This serves as an instruction for the programmer when they design the handlers for these effects. 

Handlers are implementations for a particular effect. Let us say you have an effect called 'Crop'. There are multiple ways to crop an image, such as rectangular, free-form, and circular. Provided that you keep the types of handlers consistent with what has been defined in the effect signature, you will be able to implement any number of crop operations.

These handlers are also extremely composable because of continuations. You can pass a value to a continuation that can invoke another handler, and so on. This creates a modular and composable style of code built on Algebraic Effect Handlers. 

\section{Examples in Code}

This section will highlight some practical examples of effect handlers in code. 

I have elected to use OCaml to represent these examples, and to write the dissertation. OCaml has a rich effects system and is a functional language with strong pattern matching and type inference capabilities. Its support for algebraic effects and handlers provides a natural foundation for expressing probabilistic computations as composable and interpretable effects. This allows the separation of model specification from the underlying inference mechanism, enabling modular experimentation with different inference strategies. 

\subsection{Basic Components of an Effect Oriented Program}

\begin{lstlisting}[language=OCaml]
effect Choose : int list -> int
\end{lstlisting}

\chapter{Conclusions}

\section{Final Reminder}

The body of your dissertation, before the references and any appendices,
\emph{must} finish by page~40. The introduction, after preliminary material,
should have started on page~1.

You may not change the dissertation format (e.g., reduce the font size, change
the margins, or reduce the line spacing from the default single spacing). Be
careful if you copy-paste packages into your document preamble from elsewhere.
Some \LaTeX{} packages, such as \texttt{fullpage} or \texttt{savetrees}, change
the margins of your document. Do not include them!

Over-length or incorrectly-formatted dissertations will not be accepted and you
would have to modify your dissertation and resubmit. You cannot assume we will
check your submission before the final deadline and if it requires resubmission
after the deadline to conform to the page and style requirements you will be
subject to the usual late penalties based on your final submission time.

% \bibliographystyle{plain}
\bibliographystyle{plainnat}
\bibliography{mybibfile}


% You may delete everything from \appendix up to \end{document} if you don't need it.
\appendix

\chapter{First appendix}

\section{First section}

Any appendices, including any required ethics information, should be included
after the references.

Markers do not have to consider appendices. Make sure that your contributions
are made clear in the main body of the dissertation (within the page limit).

\chapter{Participants' information sheet}

If you had human participants, include key information that they were given in
an appendix, and point to it from the ethics declaration.

\chapter{Participants' consent form}

If you had human participants, include information about how consent was
gathered in an appendix, and point to it from the ethics declaration.
This information is often a copy of a consent form.


\end{document}
